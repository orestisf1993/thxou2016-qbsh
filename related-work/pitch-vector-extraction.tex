\subsection{Pitch Vector Extraction}
Όπως \hyperref[item:pitch-vector]{αναφέρθηκε} προηγουμένως το dataset του Jang
(\cite{jang-dataset}) παρέχει αρχεία \texttt{.pv} για τα \texttt{.wav} queries.
Ωστόσο, αυτό δεν ισχύει για τα ground truth midi files και το IOACAS dataset
(\cite{IOACAS-dataset}). Επίσης, ένα ολοκληρωμένο σύστημα QbSH δεν μπορεί να
βασιστεί στην χειροκίνητη δημιουργία αυτού το αρχείου από τον χρήστη. Για αυτό
το λόγο, όπως περιγράφεται στα
\cite{ryynanen2008query, guo2012query, wang2012query, guo2013query, park2015query}
και \cite[Chapter~7]{jang2011audio}, εξάγονται τα pitch vectors από τα αρχεία
ήχου. Το block "pitch extraction" φαίνεται στα διαγράμματα
\hyperref[fig:ryynanen2008query]{\ref{fig:ryynanen2008query}}
\hyperref[fig:many-lsb-blocks]{\ref{fig:many-lsb-blocks}}
και \hyperref[fig:park2015query]{\ref{fig:park2015query}}.
Αυτή η διαδικασία μπορεί να γίνει για παράδειγμα με το
\label{pve:SAP}\href{http://mirlab.org/jang/matlab/toolbox/sap/}{Speech and Audio Processing Toolbox του Jang}
ή με άλλες τεχνικές όπως αυτές που περιγράφονται για παράδειγμα στο
\cite{ryynanen2008query}.
