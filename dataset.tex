\section{Datasets}

Θα χρησιμοποιηθούν τα 2 datasets που προσφέρονται από την σελίδα του MIREX \cite{mirex}.

\subsection{MIR-QBSH του Roger Jang}
Το πρώτο dataset που θα χρησιμοποιηθεί είναι το "Jang Dataset" \cite{jang-dataset} από τον διαγωνισμό του 2016 (ωστόσο το dataset παραμένει ίδιο από το 2009). Το dataset περιέχει:
\begin{itemize}
\item Έναν φάκελο \texttt{midiFile}:
\begin{itemize}
\item Έχει 48 αποσπάσματα από τραγούδια μικρού μεγέθους τα περισσότερα (περίπου 30s) σε μορφή MIDI\cite{midi1996complete}.
Αναφέρεται ότι πρωτόκολλο MIDI δεν μεταδίδει ηχητικό σήμα, αλλά μηνύματα που περιέχουν πληροφορίες σχετικά με το τονικό ύψος και την ένταση μιας νότας, καθώς επίσης και σήμα χρονισμού που προσδιορίζει την ταχύτητα - το tempo - ενός κομματιού.
\item To \texttt{songlist.txt} περιέχει 4 στήλες: Όνομα Αρχείου - Αγγλικό Όνομα - Κινέζικο Όνομα - Αριθμός wav αρχείων που έχουν ηχογραφηθεί γι αυτό το τραγούδι.
\end{itemize}

\item Έναν φάκελο \texttt{waveFile}:
\begin{itemize}
\item Αποτελείται από 4431 queries και 195 subjects.
\item Ο διαχωρισμός των αρχείων σε φακέλους γίνεται με την εξής ιεραρχία:
\(\text{Χρονιά} \rightarrow \text{Άτομο} \rightarrow \text{Ηχογραφήσεις}\).
\item Οι ηχογραφήσεις ανάλογα με το όνομα του \texttt{.wav} αναφέρονται στο αντίστοιχο \texttt{.midi} file που θεωρούμε ground truth.
Δηλαδή, το \texttt{00012.wav} ενός ατόμου είναι μελωδία από το τραγούδι \texttt{00012.midi} που βρίσκεται στον φάκελο \texttt{midiFile}.
\item Εκτός από τα \texttt{.wav} αρχεία υπάρχουν και αρχεία \texttt{.pv}.\label{item:pitch-vector}
Αυτά είναι αρχεία που τα έφτιαξαν οι φοιτητές που έκαναν και την αντίστοιχη
ηχογράφηση του \texttt{.wav} και περιέχουν τον τόνο (pitch) της ηχογράφησης. Αυτά δεδομένου ότι
έγιναν χειροκίνητα καλό είναι να μην θεωρούμε ότι είναι απόλυτα σωστά. Οι μετρήσεις του
pitch έγιναν με frame size 256 και μηδενικό overlap.
\item Τα \texttt{.wav} files είναι στα 8KHz, 8bit, mono
\item Όλες οι ηχογραφήσεις αρχίζουν από την αρχή του τραγουδιού και διαρκούν 8 δευτερόλεπτα.
\end{itemize}
\end{itemize}

\subsection{Institute of Acoustics, Chinese Academy of Sciences (IOACAS)}
Το δεύτερο dataset που θα χρησιμοποιηθεί είναι το "Ioacas Dataset" \cite{IOACAS-dataset} από τον διαγωνισμό του 2016. Το dataset έχει τα εξής χαρακτηριστικά:
\begin{itemize}
\item Το directory \texttt{midFile} περιέχει 298 μονοφωνικά MIDI files, το όνομα τους είναι το id του MIDI file.
\item Έχουμε 759 ηχογραφημένα \texttt{.wav} files.
\item To \texttt{query.list} αντιστοιχίζει τις ηχογραφήσεις \texttt{.wav} στα ground truth MIDI files.
\item Εδώ δεν έχουμε pitch vector files που να αντιστοιχίζονται στα \texttt{.wav} queries των ηχογραφήσεων.
Μπορεί να χρειαστεί να πραγματοποιηθεί pitch extraction όπως στο \cite{park2015query}.
\item Όλες οι ηχογραφήσεις έχουν Window 16-bit PCM format με 8k sample rate.
\item Δεν υπάρχει εγγύηση ότι οι ηχογραφήσεις αρχίζουν από την αρχή του τραγουδιού.
\end{itemize}

\subsection{Βάση Δεδομένων του Essen}
Κάποια δεδομένα από τη βάση δεδομένων του Essen\cite{esac-data} χρησιμοποιούνται
 ως dataset θορύβου, δηλαδή έχουμε κάποια επιπλέον MIDI files τα οποία δεν θα
 γίνουν query ποτέ, απλά υπάρχουν για να κάνουν μεγαλύτερη τη βάση δεδομένων. 
