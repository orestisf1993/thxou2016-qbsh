\section{Πιθανά εργαλεία που θα χρησιμοποιηθούν}
Καθώς δεν είναι δυνατή ο ακριβής προσδιορισμός των εργαλείων που θα χρησιμοποιηθούν για την επίλυση του προβλήματος τα παρακάτω αποτελούν μια αρχική προσέγγιση και πιθανών να αλλάξουν:
\begin{enumerate}
    \item \href{https://www.mathworks.com/products/matlab/}{MATLAB}
    Χρήσιμο για την πληθώρα εργαλείων σε audio processing και όχι μόνο.
    Πέρα από τα υποπροϊόντα του MATLAB μπορεί να χρησιμοποιηθούν και τα εξής:
    \begin{itemize}
        \item \href{https://www.jyu.fi/hum/laitokset/musiikki/en/research/coe/materials/mirtoolbox}{MIR toolbox}:
        Toolbox με προσανατολισμό την εξαγωγή features από αρχεία ήχου.
        \item \href{http://mirlab.org/jang/matlab/toolbox/sap/}{Speech and Audio Processing Toolbox (SAP) του Jang}:
        Toolbox που παρέχεται δωρεάν από τον Roger Jang.
        Εκτός από το SAP, στην σελίδα του παρέχεται ένα χρήσιμο on-line tutorial για επεξεργασία σημάτων ήχων \cite{jang2011audio}.
        Η χρήση αυτού του toolbox έχει ήδη αναφερθεί σε \hyperref[pve:SAP]{προηγούμενο κεφάλαιο}.
    \end{itemize}

    \item \href{https://www.python.org/}{python}:
    Γλώσσα συμβατή με πολλά λειτουργικά συστήματα που διευκολύνει την ανάπτυξη κώδικα σε σύντομο χρονικό διάστημα.
    Η python έχει ευρεία χρήση στην επιστημονική κοινότητα και για αυτό έχουν δημιουργηθεί πολλές open-source βιβλιοθήκες και project σχετικές με την επεξεργασία ήχου που μπορούν να βοηθήσουν στην εργασία μας:
    \begin{itemize}
        \item \href{https://www.scipy.org/}{SciPy}:
        Open-source οικοσύστημα για μαθηματικούς, επιστήμονες και μηχανικούς.
        Πολύ χρήσιμα μέλη του αποτελούν:
        \begin{itemize}
        \item \href{https://www.scipy.org/scipylib/index.html}{SciPy library}
        Βιβλιοθήκη που παρέχει ρουτίνες και αλγορίθμους για σύνθετες επιστημονικές διαδικασίες.
        \item \href{http://www.numpy.org/}{NumPy}:
        Βιβλιοθήκη που αποσκοπεί στην επίλυση μαθηματικών προβλημάτων.
        \item \href{http://matplotlib.org/}{matplotlib}:
        Για τη δημιουργία 2D γραφημάτων.
        \item \href{http://pandas.pydata.org/}{pandas}:
        Βιβλιοθήκη για επεξεργασία και ανάλυση δεδομένων.
        Προσφέρει συναρτήσεις και δομές δεδομένων για την επεξεργασία πινάκων και χρονοσειρών.
        \item \href{http://scikit-learn.org/stable/}{scikit-learn}:
        Βιβλιοθήκη Machine Learning.
        Ιδιαίτερα χρήσιμη θα μπορούσε να φανεί η υλοποίηση για \href{http://scikit-learn.org/stable/modules/generated/sklearn.neighbors.LSHForest.html#sklearn.neighbors.LSHForest}{lsh forest}.
        \end{itemize}

        \item Διάφορες υλοποιήσεις για \hyperref[sub:lsh]{Locality-Sensitive Hashing} όπως:
        \begin{itemize}
        \item \href{https://github.com/kayzhu/LSHash}{LSHash}
        \blockquote{A fast Python implementation of locality sensitive hashing.}
        \item \href{https://github.com/ekzhu/datasketch}{datasketch}
        \blockquote{datasketch gives you probabilistic data structures that can process vary large amount of data super fast, with little loss of accuracy.}
        \item \href{https://github.com/pixelogik/NearPy}{NearPy}
        \blockquote{NearPy is a Python framework for fast (approximated) nearest neighbour search in high dimensional vector spaces using different locality-sensitive hashing methods.}
        \end{itemize}

        \item \href{https://github.com/worldveil/dejavu}{dejavu}
        \blockquote{Audio fingerprinting and recognition in Python}

        \item \href{https://github.com/tyiannak/pyAudioAnalysis}{pyAudioAnalysis}
        \blockquote{an open-source Python library that provides a wide range of audio analysis procedures including: feature extraction, classification of audio signals, supervised and unsupervised segmentation and content visualization.}
    \end{itemize}

    \item \href{http://sonicvisualiser.org/}{sonic visualizer}:
    Γενικά χρήσιμο εργαλείο για την ανάλυση των ηχητικών κομματιών που βρίσκονται στο dataset μας.

    \item C / C++.
    Καθώς οι MATLAB και python αποτελούν γλώσσες που πολλές φορές οδηγούν σε πιο αργές λύσεις ένα σύστημα QbSH μπορεί να ωφεληθεί από τη χρήση γρηγορότερων γλωσσών χαμηλότερων επιπέδου.
    Ιδιαίτερα σε συστήματα με μεγάλες βάσεις και πολλούς χρήσεις μπορεί να αξιοποιηθεί η υπολογιστική ισχύς των GPGPU πχ μέσω της
    \href{http://www.nvidia.com/object/cuda_home_new.html}{CUDA}.
    Ωστόσο, θεωρούμε ότι η βελτίωση του συστήματος QbSH υπολογιστικά είναι εκτός του πλαισίου αυτής της εργασίας.
\end{enumerate}
