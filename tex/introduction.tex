\section{Εισαγωγή}
Η εποχή που διανύουμε είναι μια εποχή που παρουσιάζεται έκρηξη του μεγέθους των
δεδομένων. Το μεγάλο πλήθος δεδομένων δημιουργεί την ανάγκη για τη δημιουργία
αποτελεσματικών τρόπων αναζήτησης μέσα σε αυτό το μεγάλο πλήθος. Αν δεν
μπορούμε να βρούμε κάτι που υπάρχει μέσα στα δεδομένα μας τότε δεν μπορούμε να
τα εκμεταλλευτούμε, συνεπώς δημιουργείται η ανάγκη για “έξυπνους” τρόπους
αναζήτησης και ανάκτησης της πληροφορίας. Πάρα πολλά από αυτά τα δεδομένα
είναι δεδομένα ήχου, μέχρι πρότινος ο βασικός τρόπος για την ανάκτηση
πληροφορίας τέτοιου τύπου γινόταν αποκλειστικά χρησιμοποιώντας τα metadata
των δεδομένων όπως το όνομα του καλλιτέχνη, το όνομα του δίσκου, η χρονιά
κυκλοφορίας κ.α. Τα τελευταία χρόνια παρατηρούμε ότι η ανάκτηση της πληροφορίας
γίνεται και με βάση το περιεχόμενο των δεδομένων. Σε αυτή την κατηγορία
προβλημάτων συγκαταλέγεται και το Query by Singing and Humming. Με απλά λόγια
προσπαθούμε να αναγνωρίσουμε ένα κομμάτι από ένα τραγούδι το οποίο τραγουδάει
ή “μουρμουρίζει” ένας χρήστης και στη συνέχεια να του επιστρέψουμε metadata
που αναφέρονται σε αυτό το τραγούδι ή/και το ίδιο το τραγούδι.
