\section{Ανάλυση Πειραματικών Αποτελεσμάτων}
Παρακάτω παρουσιάζονται τα αποτελέσματα των πειραμάτων που διεξήχθησαν καθώς και η προεπεξεργασία
των δεδομένων που έγινε σε κάθε πείραμα.
Θα κωδικοποιήσουμε τις τεχνικές που χρησιμοποιήθηκαν για την προεπεξεργασία ώστε να μην φλυαρούμε χωρίς
να υπάρχει ανάγκη.
\begin{labeling}{codes}
  \item[Π.1] Αφαίρεση των μηδενικών από την αρχή και το τέλος της χρονοσειράς
  \item[Π.2] Μηδενισμός του pitch, στα frames που το pitch ξεφεύγει κατά T1 semitones από τη Μέση Τιμή της χρονοσειράς
  \item[Π.3] Αλλαγή του pitch αν έχουμε διαφορά πάνω από Τ2 semitones από frame σε frame (Faulty)
  \item[Π.4] Αντικατάσταση των μηδενικών pitches με το πρώτο προηγούμενο που είναι μη μηδενικό
  \item[Π.5] Φίλτρο Moving Average(9) για εξομάλυνση της χρονοσειράς
  \item[Π.6] Φίλτρο Median(9) για εξομάλυνση της χρονοσειράς
  \item[Π.7] Αφαίρεση της Μέσης Τιμής της χρονοσειράς του template(MIDI αρχείο) από τη Μέση Τιμή της χρονοσειράς του query και στη συνέχεια αφαίρεση αυτής της διαφοράς από κάθε τιμή της χρονοσειράς του query 
\end{labeling}

\subsection{Πείραμα 1}
Ακτίνα του FastDTW r = 1, Κωδικοί Προεπεξεργασίας: Π.1, Π.2, Π.3, Π.4, Π.5, Π.7
\begin{labeling}{results1}
  \item \textbf{[Top10 Hit Rate:]} 1652/4431
  \item \textbf{[Ακρίβεια:]} 37.28\%
\end{labeling}
Το πρόβλημα σε αυτό το πείραμα ήταν η χρήση του Π.3 αφού ανακαλύψαμε ότι δεν ήταν σωστά υλοποιημένο και δημιουργούσε προβλήματα.
\textbf{Όνομα αρχείου καταγραφής:} thxou\_log\_experiment\_1.log

\subsection{Πείραμα 2}
Ακτίνα του FastDTW r = 1, Κωδικοί Προεπεξεργασίας: Π.1, Π.2, Π.4, Π.7
\begin{labeling}{results2}
  \item \textbf{[Top10 Hit Rate:]} 2539/4431
  \item \textbf{[Ακρίβεια:]} 57.30\%
\end{labeling}
Εδώ αφαιρώντας το προβληματικό Π.3 και χωρίς να κάνουμε εξομάλυνση της χρονοσειράς πετύχαμε ένα αρκετά καλό αποτέλεσμα.
\textbf{Όνομα αρχείου καταγραφής:} thxou\_log\_experiment\_2.log

\subsection{Πείραμα 3}
Ακτίνα του FastDTW r = 2, Κωδικοί Προεπεξεργασίας: Π.1, Π.2, Π.4, Π.7
\begin{labeling}{results3}
  \item \textbf{[Top10 Hit Rate:]} 2604/4431
  \item \textbf{[Ακρίβεια:]} 58.77\%
\end{labeling}
Εδώ παρατηρούμε ότι με μια μικρή αύξηση της ακτίνας του FastDTW έχουμε και βελτίωση στην ακρίβειά μας.
Βέβαια υπήρξε και αύξηση του χρόνου εκτέλεσης.
\textbf{Όνομα αρχείου καταγραφής:} thxou\_log\_experiment\_3.log

\subsection{Πείραμα 4}
Ακτίνα του FastDTW r = 2, Κωδικοί Προεπεξεργασίας: Π.1, Π.2, Π.4, Π.6, Π.7
\begin{labeling}{results4}
  \item \textbf{[Top10 Hit Rate:]} 2271/4431
  \item \textbf{[Ακρίβεια:]} 51.25\%
\end{labeling}
Εδώ παρατηρούμε ότι η προσθήκη του φίλτρου Median(9) με σκοπό να εξομαλύνει την χρονοσειρά δεν βελτίωσε την κατάσταση, αντιθέτως παίρνουμε χειρότερα αποτελέσματα.
\textbf{Όνομα αρχείου καταγραφής:} thxou\_log\_experiment\_4.log

\subsection{Πείραμα 5}
Ακτίνα του FastDTW r = 2, Κωδικοί Προεπεξεργασίας: Π.1, Π.2, Π.4
\begin{labeling}{results5}
  \item \textbf{[Top10 Hit Rate:]} 309/2002
  \item \textbf{[Ακρίβεια:]} 15.43\%
\end{labeling}
Σε αυτό το πείραμα αφαιρέσαμε το φίλτρο Median(9) (Π.6) και για να δούμε αν θα υπάρξει διαφορά αφαιρέσαμε και το Π.7. Παρατηρούμε ότι τα αποτελέσματα ήταν πολύ κακά και γι αυτο το πείραμα τερματίστηκε πριν τρέξει για όλο το dataset.
\textbf{Όνομα αρχείου καταγραφής:} thxou\_log\_experiment\_5.log
