\subsection{Machine Learning Approach}
Μια διαφορετική υλοποίηση που έχουμε να προτείνουμε είναι μια προσέγγιση με τεχνικές Μηχανικής μάθησης. Όπως αναφέρεται και στο %[https://el.wikipedia.org/wiki/Μηχανική_μάθηση]
 η Μηχανική μάθηση είναι υποπεδίο της επιστήμης των υπολογιστών που αναπτύχθηκε από τη μελέτη της αναγνώρισης προτύπων και της  υπολογιστικής θεωρίας μάθησης στην τεχνητή νοημοσύνη. Πιο συγκεκριμένα κάνουμε χρήση επιβλεπόμενης μάθησης (supervised learning) κατά την οποία το υπολογιστικό πρόγραμμα δέχεται τις παραδειγματικές εισόδους καθώς και τα επιθυμητά αποτελέσματα από έναν «δάσκαλο», και ο στόχος είναι να μάθει έναν γενικό κανόνα προκειμένου να αντιστοιχίσει τις εισόδους με τα αποτελέσματα.
Πιο αναλυτικά:
\begin{itemize}
  \item \textbf{Παραδειγματικοί είσοδοι:} Αρχεία Wav των 8 KHz, 8 bit, mono των 8s εκτελεσμένα από διάφορους χρήστες από τη βάση MIR-QBSH-corpus. %TODO put link
  \item \textbf{Χαρακτηριστικά (features):} Pitch vectors μέσω της μεθόδου Cepstral [http://note.sonots.com/SciSoftware/Pitch.html] με μήκος παραθύρου 1s και μήκος επικάλυψης 0.5s.
  \item \textbf{Επιθυμητά αποτελέσματα:} Τα labels των παραπάνω τραγουδιών, δηλαδή το όνομα του καθενός.
\end{itemize}

Σημαντικό είναι να πούμε ότι χρησιμοποιήσαμε μικρό πλήθος τραγουδιών δηλαδή 9 λόγω του μικρού μεγέθους των δειγμάτων που είχαμε και σε σχέση με αυτό των features που επιλέξαμε.
Στην αρχή χρησιμοποιήθηκαν μοντέλα μηχανικής εκμάθησης όπως 
\begin{itemize}
  \item Δέντρα Απόφασης (Copmlex Trees) - Overall Accuracy 55.3\%  
  \item k-Nearest Neighbors Algorithm 	- Overall Accuracy 54.7\%
  \item Cubic Support Vector Machines	- Overall Accuracy 67.9\%
\end{itemize}
