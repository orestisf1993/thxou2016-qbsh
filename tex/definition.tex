\section{Ορισμός του προβλήματος}
Όπως αναφέρθηκε και στην Εισαγωγή το πρόβλημα με το οποίο θα ασχοληθούμε σε
αυτή την εργασία είναι το Query by Singing and Humming, δηλαδή δοθέντος ενός
ηχογραφημένου ήχου που προέκυψε από τραγούδι ή “μουρμουρητό” ένα σύστημα QbSH
πρέπει να αναγνωρίζει το μουσικό κομμάτι.

\subsection{Προκλήσεις}
Γρήγορα γίνεται αντιληπτό ότι η υλοποίηση ενός τέτοιου συστήματος παρουσιάζει
ιδιαίτερες προκλήσεις για διάφορους λόγους. Αρχικά, το σύστημα δέχεται ως
είσοδο μουσική πληροφορία η οποία παράγεται από έναν απλό χρήστη ο οποίος δεν
έχει ούτε τα μέσα, ούτε (συνήθως) τη μουσική τεχνική για να αποδώσει σωστά το
τραγούδι που θέλει να βρει. Έτσι η πληροφορία αυτή που παράγεται βρίθει
μουσικών λαθών όπως λάθη στον τόνο της φωνής, λάθη στη διάρκεια των νότων,
λάθη στο μουσικό κλειδί και στον ρυθμό. Επίσης, λόγω του ότι δεν μιλάμε για
μια επαγγελματική ηχογράφηση, η πληροφορία στην είσοδο θα έχει θόρυβο και
παραμόρφωση που δυσχεραίνουν ακόμη περισσότερο το έργο μας. Θα πρέπει λοιπόν
το σύστημα QbSH που θα υλοποιηθεί να παρουσιάζει ευρωστία στις καταστάσεις που
αναφέρθηκαν παραπάνω.
\newline
Άλλη μια πρόκληση που παρουσιάζει το πρόβλημα του QbSH είναι η αναζήτηση στη
βάση δεδομένων. Όπως αναφέρθηκε και στην εισαγωγή πρέπει να γίνει μια σύγκριση
 της πληροφορίας της εισόδου με μελωδίες που βρίσκονται σε μια βάση δεδομένων.
 Είναι αντιληπτό ότι για μια εμπορική εφαρμογή μια τέτοια βάση δεδομένων θα
 έχει ένα πολύ μεγάλο πλήθος δεδομένων το οποίο σημαίνει ότι η αποθήκευση των
  δεδομένων θα πρέπει να γίνεται με τέτοιο τρόπο ώστε να μπορεί το σύστημα
  να συγκρίνει σχετικά γρήγορα την είσοδο με τα δεδομένα στη βάση.
\subsection{Ανάλυση του προβλήματος}
Το πρόβλημα του QbSH μπορεί να χωριστεί σε τρία υποπροβλήματα. Πρώτο
υποπρόβλημα είναι η δημιουργία της βάσης δεδομένων που περιέχει τις μελωδίες,
κάθε μελωδία έχει μια ετικέτα που ορίζει από ποιο τραγούδι προέρχεται αυτή η
μελωδία, αυτές οι μελωδίες κατά τη διάρκεια της λειτουργίας του συστήματος θα
συγκρίνονται με την είσοδο. Οι μελωδίες αυτές δημιουργούνται είτε χειροκίνητα
και στη συνέχεια καταχωρούνται στη βάση δεδομένων, είτε εξάγονται από τραγούδια
με αυτοματοποιημένο τρόπο (αυτό είναι ένα άλλο αρκετά δύσκολο πρόβλημα).
Τα αρχεία αυτά των μελωδιών μπορεί να είναι αρχεία ήχου (π.χ. wav) ή αρχεία
MIDI πολυφωνικά ή μονοφωνικά. Στη συνέχεια κάνουμε εξαγωγή ακουστικών
χαρακτηριστικών από τις μελωδίες ώστε να μπορεί να γίνει η σύγκριση με την
είσοδο του χρήστη.
\newline
Το δεύτερο υποπρόβλημα είναι η επεξεργασία της εισόδου που δίνει ο χρήστης η
οποία όπως αναφέρθηκε παραπάνω παρουσιάζει πολλά προβλήματα. Πρέπει να γίνει
επεξεργασία ώστε να αφαιρεθεί ο θόρυβος και να βρεθούν ακουστικά χαρακτηριστικά
με τέτοιο τρόπο ώστε να μην υπάρχει πρόβλημα λόγω των μουσικών λαθών που έχουν
γίνει από τον χρήστη.
\newline
Το τρίτο υποπρόβλημα είναι η αντιστοίχιση της εισόδου με μελωδίες που βρίσκονται
στη βάση δεδομένων ώστε να επιστραφεί στον χρήστη μια λίστα με τα τραγούδια που
μοιάζουν περισσότερο με την είσοδο που έδωσε στο σύστημα.
