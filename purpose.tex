\section{Σκοπός της Εργασίας}
Σύμφωνα με όσα αναφέρθηκαν παραπάνω είναι αντιληπτό ότι έχουμε να κάνουμε με
ένα αρκετά μεγάλο και πολύπλοκο πρόβλημα το οποίο δεν μπορεί να επιλυθεί πλήρως
και σε επίπεδο εμπορικής εφαρμογής στα πλαίσια αυτής της εργασίας. Κατά συνέπεια
η προσέγγισή μας στο πρόβλημα θα γίνει με βάση τους περιορισμούς που θέτει το
πρόβλημα του Query by Singing/Humming του MIREX 2016\cite{mirex}. Το πρόβλημα
του MIREX έχει δύο υποπροβλήματα, η διαφορά τους έγκειται στις μελωδίες που
περιέχονται στη βάση. Παρακάτω γίνεται ανάλυση των δύο υποπροβλημάτων σε
επίπεδο εισόδου χρήστη - βάσης δεδομένων και εξόδου του συστήματος.

\subsection{Είσοδος Χρήστη:}
Και για τα δύο υποπροβλήματα τα dataset των εισόδων είναι τα ίδια, είναι τα
dataset MIR-QbSH και IOACAS (περισσότερα για αυτά τα datasets αναφέρονται
παρακάτω) τα οποία περιέχουν ανθρώπινα μουρμουρητά/τραγούδια σε μορφή .wav.

\subsection{Βάση Δεδομένων:}
\begin{itemize}
  \item Για το πρώτο υποπρόβλημα έχουμε για ground truth, MIDI files 48 από το
  MIR-QbSH, 106 από το IOACAS και 2000+ από μια καθαρή version της βάσης
  δεδομένων του Essen \cite{esac-dataset}
  \item Για το δεύτερο υποπρόβλημα έχουμε .wav files από όλα τα dataset με
  εξαίρεση την είσοδο του χρήστη.
\end{itemize}


\subsection{Έξοδος και Αξιολόγηση:}
Η έξοδος είναι η ίδια και για τα δύο προβλήματα, βγάζουμε μια λίστα με τα 10
τραγούδια τα οποία πιστεύουμε ότι μοιάζουν περισσότερο με το τραγούδι που έδωσε
ο χρήστης. Αν το τραγούδι που έδωσε ο χρήστης είναι στη λίστα που έβγαλε το
σύστημα μας παίρνουμε 1 πόντο, αλλιώς παίρνουμε 0 πόντους.
